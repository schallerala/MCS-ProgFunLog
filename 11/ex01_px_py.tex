\documentclass[border=10pt]{standalone}

\usepackage[edges]{forest}
\tikzset{
    disabled/.style = { fill=gray, text=white }
}

% To be able to expand command in forest
\bracketset{action character=@} % https://tex.stackexchange.com/questions/244067/expansion-in-forest


\newcommand{\unify}[2]{%
{#2}, edge label={node[midway,left] {#1}}}

\newcommand{\cutbranch}[1]{%
#1, disabled, edge={dashed, gray}, edge label={node[midway, fill=white] {X}}}

\begin{document}

% p(1).
% p(2) :- !.
% p(3).
%
% ?- p(X), p(Y).
\begin{forest}
for tree={edge={-latex},l+=0.7cm,s sep=1.4cm}
[{p(X), p(Y)}
    [@+\unify{X=1}{p(1), p(Y)}
        [@+\unify{Y=1}{p(1)} [Success]]
        [@+\unify{Y=2}{p(2), !} [! [Success]]]
        [\cutbranch{p(3)}]
    ]
    [@+\unify{X=2}{p(2), !, p(Y)}
        [{!, p(Y)}
            [p(Y)
                [@+\unify{Y=1}{p(1)} [Success]]
                [@+\unify{Y=2}{p(2), !} [! [Success]]]
                [@+\cutbranch{p(3)}]
            ]
        ]
    ]
    [@+\cutbranch{{p(3), p(Y)}}]
]
\end{forest}

\end{document}