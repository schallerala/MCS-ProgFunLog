\documentclass[12pt]{article}

\usepackage[
    a4paper,
    hmargin=2cm, vmargin=1.3cm,
    headheight=38pt, % as per the warning by fancyhdr
    includehead, includefoot
]{geometry}

\usepackage[english]{babel}
\usepackage[utf8x]{inputenc}
\usepackage[T1]{fontenc}

\usepackage{fancyhdr} % extensice control of page headers and footers
\usepackage{graphicx}
\usepackage{parskip} % add space between paragraphs and remove first line's indent
\usepackage[dvipsnames]{xcolor} % color the text for the color code: https://en.wikibooks.org/wiki/LaTeX/Colors


\usepackage{enumitem}

% ---------------- SETUP HEADER & FOOTER ----------------------
\pagestyle{fancy}
% \fancyhf{} % sets both header and footer to nothing
% \renewcommand{\headrulewidth}{0pt} % remove headers bottom line

\def\code#1{\texttt{#1}}

\newcommand{\unilogo}[1][0.16\textwidth]{\includegraphics[width=#1]{logo_uni}}

\newcommand*{\fullref}[1]{\hyperref[{#1}]{\autoref*{#1} \nameref*{#1}}} % https://tex.stackexchange.com/a/121871

\newcommand{\commentcolor}{\color{Gray}}


\newcounter{question}
\newcommand{\questioncolor}{\color{MidnightBlue}}
\def\newquestion#1{
    \refstepcounter{question}
    {
        \textbf{Exercise~\thequestion:}
        \questioncolor
        \emph{#1}
    }
}
\def\question#1{
    {
        \questioncolor
        \emph{#1}
    }
}

\makeatletter
\newcommand{\prof}[1]{\def\@prof{#1}}
\newcommand{\theprof}{\@ifundefined{@prof}{}{\\Prof.: \@prof}}
\newcommand{\student}[1]{\def\@student{#1}}
\newcommand{\thestudent}{\@ifundefined{@student}{}{\@student}}
\newcommand{\seriesnumber}[1]{\def\@seriesnumber{#1}}
\newcommand{\theseriesnumber}{\@ifundefined{@seriesnumber}{}{Series \@seriesnumber}}
\makeatother

\title{\vspace{-2cm}\unilogo[0.3\textwidth]\\[0.9cm]Série 1 -- Programmation fonctionnelle\\Haskell}
\student{Alain Schaller \& Yael Iseli\\\texttt{16-896-375} \& \texttt{14-8215-24}}
\prof{Stéphane Le Peutrec}
\seriesnumber{1}
\author{\thestudent\theprof}

\lhead{\unilogo}
\rhead{\thestudent\\\theseriesnumber}

% ---------------------- DOCUMENT ----------------------------
\begin{document}

% \textsc{\LARGE Secure code review}\\%[0.2em]
%\large{\color{gray} ...}\\

\maketitle

%
%
% ---------------------------------
\section*{Introduction}

\setcounter{question}{1}
\newquestion{GHCi + tutoriel}

\begin{enumerate}[label=\questioncolor\alph*)]
    \item \question{Prenez connaissance des diverses commandes prédéfinies, (accessibles depuis la commande
    \texttt{:help}), essayez entre autres la commande \texttt{:t} et la commande \texttt{:set prompt}}



    \begin{description}
        \item[\questioncolor\texttt{:t (==)}]
        \item[\questioncolor\texttt{:i (==)}]
        \item[\questioncolor\texttt{:t max}]
    \end{description}

\end{enumerate}

\newquestion{Create yours first functions}

\begin{enumerate}[label=\questioncolor\alph*)]
    \item \question{A function that calculates the sum of 4 parameters of type \texttt{Int}}



    \item \question{A function \texttt{max3} that calculates the maximal value of 3 parameters of type \texttt{Int}. Implement 5 versions of this function : at least one using if, at least one using the guards and at least one using where. You can use the predefined function max.}



    \item \question{A function \texttt{sign} with a parameter of type \texttt{Int} and \texttt{returns} the string "this number is positive" if the
    number is positive, "this number is negative" if the number is negative and "this number is \texttt{null}" if the number is $0$. Submit a version with if, a version with guard.\\\commentcolor---\\
    Note: The ++ operator appends strings.}

\end{enumerate}


\end{document}